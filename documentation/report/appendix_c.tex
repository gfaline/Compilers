\section{Appendix C: Testing Scripts}

\subsection{Main testing framework (\texttt{testutil.sh})}

\begin{lstlisting}[language=bash,backgroundcolor=\color{backgroundcolor}]
TheCompiler=$(realpath $(dirname "$0")/../propeller.native)
TheWrapper=$(realpath $(dirname "$0")/../prc.sh)

# CheckExpectWReturnCode Exec ExpectedOutput ExpectedStderr ExpectedReturnCode
# Exec: command to execute
# ExpectedOutput: Reference output. Pass "IGNORED" to ignore all output.
# ExpectedStderr: Reference output of stderr. Pass "IGNORED" to ignore all output to stderr.
# ExpectedReturnCode: Expected return code. Pass "FAIL" to accept any non-zero return code.
CheckExpectWReturnCode() {
  RETCODE=0
  BADRET=0
  EXEC=$1
  ExpectedOutput=$2
  ExpectedStderr=$3
  ExpectedRet=$4
  OutputT=$(mktemp prtest.outt.XXXXX)
  StderrT=$(mktemp prtest.errt.XXXXX)
  Output=$(mktemp prtest.out.XXXXX)
  Stderr=$(mktemp prtest.err.XXXXX)

  eval "${EXEC} > ${OutputT} 2> ${StderrT}"
  Ret=$?
  tr -d '\015' < ${OutputT} > ${Output}
  tr -d '\015' < ${StderrT} > ${Stderr}

  [[ ${ExpectedOutput} == IGNORED ]] || diff -u ${ExpectedOutput} ${Output} || RETCODE=1
  [[ ${ExpectedStderr} == IGNORED ]] || diff -u ${ExpectedStderr} ${Stderr} || RETCODE=1

  if [[ ${ExpectedRet} == FAIL ]]; then
    [ $Ret -ne 0 ] || BADRET=1
  else
    [ $Ret -eq $ExpectedRet ] || BADRET=1
  fi

  if [[ $BADRET == 1 ]]; then
    RETCODE=1
    echo "unexpected return code $Ret"
  fi
  
  rm ${OutputT} ${StderrT} ${Output} ${Stderr}
  return $RETCODE
}

# CheckExpect Exec ExpectedOutput
# convenience function
CheckExpect() {
  EXEC=$1
  ExpectedOutput=$2
  CheckExpectWReturnCode "$EXEC" $ExpectedOutput IGNORED 0
}

# CheckFail Exec ExpectedStderr
# convenience function
CheckFail() {
  EXEC=$1
  ExpectedStderr=$2
  CheckExpectWReturnCode "$EXEC" IGNORED $ExpectedStderr FAIL
}


# CompiledCheckExpectWReturnCode Source ExpectedOutput ExpectedStderr ExpectedReturnCode
# convenience function. Compile the given propeller source program and run it with
# CheckExpectedWReturnCode
CompiledCheckExpectWReturnCode() {
  SRC=$1
  shift
  BASENAME=${SRC%.pr}
  EXEC=$BASENAME.out
  [[ $UNAME =~ (CYGWIN|MINGW|MSYS).* ]] && EXEC=$BASENAME.exe
  ${TheWrapper} $SRC
  CheckExpectWReturnCode "./$EXEC" $@
  RET=$?
  rm $EXEC
  return $RET
}

# CompiledCheckFail Exec ExpectedStderr
# convenience function
CompiledCheckExpect() {
  SRC=$1
  ExpectedOutput=$2
  CompiledCheckExpectWReturnCode $SRC $ExpectedOutput IGNORED 0
}

# CompiledCheckFail Exec ExpectedStderr
# convenience function
CompiledCheckFail() {
  SRC=$1
  ExpectedStderr=$2
  CompiledCheckExpectWReturnCode $SRC IGNORED $ExpectedStderr FAIL
}
\end{lstlisting}

\subsection{The "Hello World" test suite (\texttt{tests-hello.sh})}

\begin{lstlisting}[language=bash,backgroundcolor=\color{backgroundcolor}]
#!/usr/bin/env bash
source $(dirname "$0")/testutil.sh
cd $(dirname "$0")/hello

RET=0
CompiledCheckExpect hello.pr hello.exp || { echo "hello.pr failed"; RET=1; }
CompiledCheckExpectWReturnCode hello_ret.pr hello.exp IGNORED 5 || { echo "hello_ret.pr failed"; RET=1; }
CheckFail "$TheWrapper hello_bad.pr" IGNORED || { echo "hello_bad.pr failed"; RET=1; }
[ $RET -eq 0 ] && echo "all 3 tests passed"
exit $RET
\end{lstlisting}

\subsection{The parser test suite (\texttt{tests-parser.sh})}

\begin{lstlisting}[language=bash,backgroundcolor=\color{backgroundcolor}]
#!/usr/bin/env bash
source $(dirname "$0")/testutil.sh
cd $(dirname "$0")/parser

for i in test-*.pr
do
  out=${i%.pr}.exp
  if CheckExpect "$TheCompiler -a $i" $out; then
    echo "$i passed"
  else
    echo "$i failed"
  fi
done

for i in fail-*.pr
do
  err=${i%.pr}.err
  if CheckFail "$TheCompiler -a $i" $err; then
    echo "$i passed"
  else
    echo "$i failed"
  fi
done
\end{lstlisting}

\subsection{The extended test suite (\texttt{tests-extended.sh})}

\begin{lstlisting}[language=bash,backgroundcolor=\color{backgroundcolor}]
#!/usr/bin/env bash
source $(dirname "$0")/testutil.sh
cd $(dirname "$0")/exttest

failed=0

for i in test-*.pr
do
  out=${i%.pr}.exp
  if CompiledCheckExpect "$i" $out; then
    echo "$i passed"
  else
    echo "$i failed"
    failed=$(($failed+1))
  fi
done

for i in fail-*.pr
do
  err=${i%.pr}.err
  if CheckFail "$TheCompiler $i" $err; then
    echo "$i passed"
  else
    echo "$i failed"
    failed=$(($failed+1))
  fi
done

if [ $failed == 0 ]
then
  echo "All tests passed."
else
  echo "$failed test(s) failed."
fi
\end{lstlisting}
