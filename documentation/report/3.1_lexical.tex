\section{Language Manual}

\subsubsection{Notion used in this Manual}

This document uses a variation of Backus-Naur form to describe the syntax of the language. The most
significant additions are the character range notation (\verb|"a"-"d"| rather than
\verb:"a"|"b"|"c"|"d":), and the notation for items repeated zero or more times
(\verb|{<characters>}| means \verb|<characters>| repeated zero or more times).

\subsubsection{Significant changes compared to the tentative version}
\begin{itemize}
\item Escape sequences are now well-defined. Single quotes cannot be escaped.
\item Strings are no longer syntactic sugar for int lists.
\item Semantics of bindings have changed. For non-\texttt{external} objects, bindings are managed
at compile time.
\item Function binding only works for local object variables. Functions bound to internal objects receive two copies
      of the new value as their arguments.
\end{itemize}

% Lexical
\subsection{Lexical Conventions}

% Comments
\subsubsection{Comments}
Propeller supports single-line comments. Any sequence of characters following a hash (\verb|#|)
that is not part of a string literal will be treated as a comment. Comments automatically terminate
at the end of a line.

\begin{mylisting}
# This is a comment.

# valid identifiers
x
ready?
aBc123
o_o
AsDfG_1234_5?

# invalid identifiers
8eight
two__underscores
propeller_
too?early
huh_?
\end{mylisting}

% Identifiers
\subsubsection{Identifiers}

\begin{verbatim}
    <letter> ::= "A"-"Z" | "a"-"z"
    <digit>  ::= "0"-"9"
  <alphnum>  ::= <letter> | <digit>
<identifier> ::= <letter> ("?" | "")
               | <letter> ("_" | "") {<alphnum> | <alphnum> "_"}
                 <alphnum> ("?" | "")
\end{verbatim}

\noindent
An identifier in Propeller is a character sequence consisting of letters, digits, underscores, and one
optional question mark. Identifiers must begin with letters, cannot contain consecutive underscores,
and cannot end with an underscore. Additionally, identifiers may end with a single question mark,
but may not end with an underscore followed by a question mark.

% Keywords
\subsubsection{Keywords}
Propeller has 24 reserved keywords:
\begin{mylisting}
and     bind     break  continue elif
else    external float  fn       for
from    if       int    list     not
objdef  or       return str      to
unbind  void     while  xor                     
\end{mylisting}

% Separators
\subsubsection{Separators}
Propeller has 10 separators used to construct literals, define functions, separate statements, and
more:
\begin{mylisting}
( ) [ ] { }
, . ;   ->
\end{mylisting}

% Operators
\subsubsection{Operators}
Propeller has 15 operators for comparison, logic, and arithmetic.
\begin{mylisting}
# comparison  logic  arithmetic
  =           and    +
  !=          or     -
  >           not    *
  <           xor    /
  >=                 %
  <=
\end{mylisting}

% Literals
\subsubsection{Literals}
\begin{verbatim}
      <int-literal> ::= {<digit>}
    <float-literal> ::= {<digit>} "." {<digit>}
     <bool-literal> ::= "true" | "false"
<string-characters> ::= character or escape sequence
   <string-literal> ::= "'" {<string-characters>} "'"
\end{verbatim}

\noindent
Supported escape sequences include \verb|\n| for linefeed, \verb|\r| for carriage return,
\verb|\t| for horizontal tabulation.

\begin{mylisting}
# int literals     float literals     boolean literals       string literals
1                  3.14               true                   'hello'
23                 0.78               false                  'h0wdy!'
0                  12.345                                    'Hello World!\n'
5839407430         0.999
\end{mylisting}
