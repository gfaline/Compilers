\subsection{Semantics}

Propeller's operational semantics is heavily inspired by C-like languages. For the sake of
brevity, commonplace semantics found in other widely-used languages are omitted in
favor of the semantics guiding Propeller's most prominent features.

% Data Types
\subsubsection{Data Types}

% Primitives
\paragraph{Primitive Types}
Primitive types use the following internal representations:
\begin{center}
\begin{tabular}{| c | c | p{8cm} | }
\hline
 \textbf{Type} & \textbf{Size (bytes)} & \textbf{Description} \\
 \hline
 \texttt{int} & 4 & Integer \\
 \hline
 \texttt{float} & 8 & IEEE 754 floating point \\
 \hline
 \texttt{bool} & 1 & Boolean \\
 \hline
 \texttt{str} & varies & Stores UTF-8 encoded characters of a
 string \\
 \hline
 \texttt{void} & N/A & Only used as return type for functions that return nothing \\
\hline
\end{tabular}
\end{center}

% Lists
\paragraph{Lists}
\texttt{list}s contain one or more elements of the same type; the elements are immutable, and can be
indexed with separators \textbf{[ ]}. Elements of a list are stored sequentially in memory.
For example, a \texttt{bool list} of length 5 takes up 5 bytes in memory, with the first element
stored in the first byte, the second element in the second byte, and so on. Attempting to index
a list outside of its bounds results in undefined behavior. Although the elements of a list are
immutable, a list variable may be overwritten by assigning another list to it.

% Objects
\paragraph{Objects}
Propeller allows users to define custom types called objects. An object has one or
more properties, which are variables of primitive types. Additionally, functions may be bound
to these properties and executed upon a change in the property's value. A runtime can have
several predefined objects from an external library, but these objects must be defined
and prefixed with the \texttt{external} keyword. Object variables may not be passed
to functions.

% Operations
\subsubsection{Operations}
\paragraph{\texttt{int} and \texttt{float} Operations}
Comparison and arithmetic operators are overloaded in Propeller. Comparing an \texttt{int} to a
\texttt{float} will result in the promotion of the \texttt{int} expression to a \texttt{float} expression;
similarly, combining an \texttt{int} and a \texttt{float} with a binary arithmetic operator will result
in the promotion of the \texttt{int} expression to a \texttt{float} expression, and the result of the
arithmetic operation will be a \texttt{float}. Additionally, the modulus \texttt{\%}
operator, only takes \texttt{int} operands, and the divison of an \texttt{int} by a \texttt{int} returns
a \texttt{float}.

The modulo operator follows the "truncate" definition.

\begin{mylisting}
# comparison
2.3 == 1;   # false
4 != 5;     # true
8.34 > 3;   # true
9 < 10;     # true
5.5 >= 5.6; # false
100 <= 100; # true

# arithmetic
4 + 3;    # 7
2.0 - 1;  # 1.0
-3.3 * 3; # -9.9
5 / 2;    # 2
5.0 / 2;  # 2.5
6 % 4;    # 2

# boolean comparison
true != false; # true
false == true; # false

# boolean operations
not false;      # true
true and false; # false
true or false;  # true
true xor true;  # false
\end{mylisting}

\noindent Lists can be indexed using the \verb|[]| operator.

\begin{mylisting}
# list indexing
int list l = [1, 2, 3];
l[1];           # 2
\end{mylisting}

% Control Flow
\subsubsection{Control Flow}
Propeller uses \texttt{if/elif/else} clauses, \texttt{for} loops, and \texttt{while} loops.
Each control flow method in its entirety is a statement, and the body of an
\texttt{if/elif/else} clause or loop is comprised of a list of statements. Control flow
semantics are C-like, with the following differences:
\begin{itemize}
    \item \texttt{if/elif/while} expressions need not be enclosed in parentheses.
    \item Statements following \texttt{if/elif/else}, \texttt{for}, and \texttt{while} must
          be enclosed in curly braces \texttt{\{ \}}.
    \item There is no such thing as a "block" - each control flow method is followed by a list
          of one or more statements enclosed in curly braces.
    \item \texttt{for} loops have a unique syntax, and are intended to execute a given number of
          times. When writing a \texttt{for} loop, the name of the looping variable must be given, along
          with integer expressions that evaluate to the looping variable's initial and final
          value, respectively. When the looping variable is greater than the final value, the
          loop terminates.
   \item Keywords \texttt{break} and \texttt{continue} can be used to jump out of a loop or continue
         to its next iteration, respectively.
\end{itemize}

% Binding
\subsubsection{Binding}

Functions can be bound to properties of objects such that whenever the property is assigned a value,
the functions bound to that property are called.

Let $\beta$ be the bindings that are currently established during execution of the program. $\beta$
is one of the environment metavariable of Propeller's operational semantics.
$\beta(o, p)$ is a set of functions bound to property $p$ of object $o$. Note that this way objects
of the same custom type don't share bindings.

A function bound to a property must accept two parameters: two values of the same type as the property
itself, passing the old value of the property and new value of the property respectively.

When multiple functions are bound to the same property of an object, their order of execution is defined by the order in which they were
bound.

Semi-formal operational semantics of syntactical forms related to binding will be given below.
$\rho(o, p)$ retrieves the location where property $p$ of object $o$ is stored, and $\sigma(l)$ is
the value at location $l$.

$$
\dfrac{\begin{gathered}
\langle e,\rho,\sigma,\beta,\cdots \rangle \Downarrow
\langle v,\rho,\sigma_0,\beta,\cdots \rangle\\
\textrm{for each } f_i\in\beta(o, p), i=1\dots n\\
\langle f_i(\sigma_0(\rho(o,p),v),\rho,\sigma_{i-1},\beta,\cdots \rangle \Downarrow
\langle void,\rho,\sigma_i,\beta,\cdots \rangle
\end{gathered}
}
{\langle \textsc{PropertyAssign}(o,p,e),\rho,\sigma,\beta,\cdots \rangle
\Downarrow
\langle v,\rho,\sigma_n\{\rho(o,p)\mapsto v\},\beta,\cdots \rangle
} \qquad \textsc{PropertyAssign}
$$

$$
\dfrac{}
{\langle \textsc{bind}(o,p,f),\rho,\sigma,\beta,\cdots \rangle
\Downarrow
\langle void,\rho,\sigma,\beta\{(o,p)\mapsto \beta(o,p)\cup \{f\}\},\cdots \rangle
} \qquad \textsc{Bind}
$$

$$
\dfrac{}
{\langle \textsc{unbind}(o,p,f),\rho,\sigma,\beta,\cdots \rangle
\Downarrow
\langle void,\rho,\sigma,\beta\{(o,p)\mapsto \beta(o,p)\setminus \{f\}\},\cdots \rangle
} \qquad \textsc{Unbind}
$$

The semantics above only applies if the object in \textsc{PropertyAssign} is defined as
\texttt{external}. For non-\texttt{external} objects, bindings are managed statically at compile
time -- they are effective for all statements that are parsed in between the \verb|bind| statement
and its corresponding \verb|unbind| statement.

Due to limitations of the implementation, bindings only work for local variables. In addition,
functions bound to non-\texttt{external} objects do not recieve the property's previous value when called. Finally, external
objects cannot have values assigned to their properties.

% Program Execution
\subsubsection{Program Execution}

When a program written in Propeller is executed, it begins from a function called \verb|init()|.
After \verb|init()| returns, it enters an event loop defined by the runtime library. For the
most basic text-mode only runtime, the event loop simply terminates the program.
