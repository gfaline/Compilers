\section{Testing}

All testing facility is located in the \texttt{tests} directory of the compiler source code.

\subsection{Example source code and generated LLVM IR}

See Appendix B.

\subsection{Test scripts}

See Appendix C. Each function is documented in the comment above it.

\subsection{Test Automation}

A 105-style testing interface (notably the name ``CheckExpect") was used. The testing interface
as well as all test scripts are written in Bash. Three different aspects
of execution result can be checked against a given standard: standard output, standard error and
return code. The first two can be ignored if needed. Since the compiler exits with non zero return
code if an invalid program is fed to it, it can be used to verify test cases in which the
compilation is expected to fail.

There are convenience wrapper functions of \verb|CheckExpectWReturnCode| provided for different
tasks to simplify the process to make a test suite.

For each test suite, a test script is created. Running these test scripts will run all the tests
in that test suite.

\subsection{Testing task split}

The current testing framework is written by Chris. Parser unit tests are provided by Gwendolyn
and Isra. All team members contributed to the extended test suite. Previously the project used
a testing script written by Gwendolyn that was based on the test script of MicroC.
