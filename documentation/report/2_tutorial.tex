\section{Quick Tutorial} 

% Minimal program
For those who have programmed in C-like languages, Propeller will feel familiar in
an instant. Here's an example of a minimal Propeller program:

\begin{mylisting}
fn init() -> int
{
  prints('Hello World!\n');
  return 0;
}
\end{mylisting}

% Statement grouping
\noindent
Like C, Propeller uses curly braces to group statements. However, unlike C,
these braces are mandatory even if the block only consists of only one statement.

\begin{mylisting}
fn foo(int n) -> int
{
  if n % 3 == 0
  { return 9; }
  else
  { return 42; }
}
\end{mylisting}

\noindent
Propeller doesn't have scoped local variables -- all locals must be declared at the beginning of a function.

\begin{mylisting}
fn foo() -> void
{
  int i;
  i = 0;
  # int invalid_a;     invalid because decalaration appeared after a statement
  while i < 10
  {
    # int invalid_b;   invalid because this is block is not the beginning of a function
    print(i);
    i = i + 1;
  }
}
\end{mylisting}

\noindent
Defining a custom object is much like defining a struct in C:
\begin{mylisting}
objdef Jumbo
{
  str name;
  int age;
  float gpa;
}
\end{mylisting}

\noindent
Similarly, accessing and assigning to properties of objects is just like accessing and assigning to fields of a struct in C:

\begin{mylisting}
Jumbo jim;

jim.name = 'Jim';
jim.age  = 24;
jim.gpa  = 3.73;
\end{mylisting}

\noindent
Functions can be bound to properties, such that these functions are called whenever
a new value is assigned to the property. Binding functions must take two arguments
whose types match the type of the property.

\begin{mylisting}
fn celebrate(int old, int new) -> void
{
  print(new);
}

...

  bind(jim.age, celebrate);
\end{mylisting}
