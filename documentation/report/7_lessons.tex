\section{Lessons learned} 

\subsection{Isra}
\begin{itemize}
    \item This class gave me a new perspective on programming and a deeper understanding of how the languages I use work. I've taken theory based classes like 170, but I didn't fully understand how what we learned about finite automata and context free grammars was applicable until now. It can be really useful to know how things work under the hood as a programmer.
    \item The scanner, parser, and semantic implementation wasn't very hard to understand, but code generation and llvm is much more complex. The documention for llvm isn't the best either. It has a steep learning curve, so it's best to start understanding codegen as soon as possible.
    \item Starting earlier would have been beneficial. I think we should have set strict deadlines about when we want the smaller features to be implemented throughout the semester, so we wouldn't have been as stressed before the larger deadlines. 
\end{itemize}

\subsection{Gwen}
\begin{itemize}
    \item Keep on track from the start. It is easy to get lost in the periods of time between assignments. It's also easy to get lost in how long things will take. I wish we were better about having smaller internal deadlines but life was very in the way for all of us. 
    \item Functional programming forces me to work a little differently
    \item Compilers were a total foreign body to me. Going through first the labs them implementation helped me understand more of this black magic. 
    \item Don't be afraid to cut what isn't working and isn't necessary. Scaling down and deciding what the most important bits are help to focus the work.
\end{itemize}

\subsection{Randy}
\begin{itemize}
    \item Functional programming is super cool, but it has its limits. We resorted to using a
          StringMap reference (which behaves like a global variable of sorts) to keep track of what
          functions are bound to the properties of object variables. I'm sure there's a
          functional way to do that, but this approach made the most sense to us.
    \item Start early, on everything! Code generation can get particularly hairy.
    \item The OCaml LLVM API can be pretty confusing, so be sure to ask your advisor a bunch
          of questions if you can't figure out how to get something working.
    \item Temper your expectations from the beginning. Your language should only have a handful of
          core features are unique and super exciting - don't promise too much!
\end{itemize}

\subsection{Chris}
\begin{itemize}
\item Even after been warned multiple times, I still have the habit of not starting the real work
until the last minute.
\item Functional programming is fun in a way that it forces me to think in completely different
patterns.
\item (As advice for future students as well) Learn to manage the expectations. As initially
planned, Propeller was clearly way too ambitious, even with GUIs, high-level list operations and
all the bells and whistles. As time flew by, some of the goals are clearly unrealistic for a
semester-long project and had to be dropped.
\item (When implementing external object bindings) If searching something gives no useful results,
and all hope is lost, trying all possible combinations could be a good way to go.
\item Test-oriented development can be beneficial, but could result in a compiler that doesn't work
with code that uses any untested features.
\item Function pointers in LLVM causes even more confusion than function pointers in C++.
\item I feel like as a team made of complete strangers, we could use more communication throughout
the project.
\end{itemize}
