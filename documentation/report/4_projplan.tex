\section{Project Plan} 

\subsection{Process used}

Our team used git for source code version control. In general, team members worked on
their tasks individually. All team members have basic knowledge of git, and had
direct access to the main repository (\url{https://github.com/gfaline/Compilers}).

The team held irregular meetings to coordinate strategies for completing each
deliverable. Initially, we used GitHub issues to delegate and assign tasks, but as the
semester progressed, we moved most activity to a Discord server. Our advisor, Mert, has
access to a special channel in this server, where we fielded several questions about
design and implementation throughout the semester.

Specification documents are can also be found in the git repository. Early drafts were created using
the \LaTeX \ collaboration platform Overleaf.

% Style guide
\subsection{Style guide}

As the members worked on their own, there is no explicit style guideline. The untold rule is to
follow the style of existing code. Most observed rules are:

\begin{itemize}
\item Indent with two spaces.
\item Each line should not exceed 100 characters (not strictly enforced).
\item Name variables descriptively, unless it's only used by a few expressions that follow it.
\item Test as soon as the code is working.
\item Put the "in" of a let statement at the end of the line if it's a variable, and on a new
      line if it's a function
\end{itemize}

% Timeline
\subsection{Project timeline}

Weekly commit history (generated with \verb|git-bars|):
{\small
\begin{verbatim}
281 commits over 10 week(s)
2022/18  39   **********************
2022/17  1    
2022/16  59   **********************************
2022/15  11   ******
2022/14  6    ***
2022/12  33   *******************
2022/09  12   *******
2022/08  85   **************************************************
2022/07  34   ********************
2022/06  1    
\end{verbatim}
}

The project is very clearly deadline-driven. Regular project check-ins
helped push the project forward, as evidenced by the spike in activity around each one.

The task list that laid out by the team at the beginning of the semester is listed below:

\begin{verbatim}
[ ] Hello World (print an integer)
 [ ]   SAST for expressions with only integer literals
 [ ]   dummy function call semantics & SAST (no type checking)
 [ ]   dummy semantics for return
 [ ]   stubs for statement sequencing and statement blocks (without typing etc)
 [ ]   code generation for function call and return
 [ ]   built-in function print_int
 [ ]   script for linkage and other shenanigans (generating the executable)

[ ] SAST for expressions
 [ ]   arithmetic operations between integers
 [ ]   arithmetic operations between floats
 [ ]   arithmetic operations between floats and integers (promotion)
 [ ]   comparison between integers and floats (promotion, can be split like above)
 [ ]   boolean operations
 [ ]   boolean comparison
 [ ]   type checking for all invalid cases

[ ] Expression & code generation (without variables)
 [ ]   code generation for integral arithmetic operators
 [ ]   code generation for float operators
 [ ]   code generation for boolean operators
 [ ]   code generation for comparison operators
 [ ]   assignment to variables: code generation
 [ ]   built-in function print_bool and print_float
 [ ]   tests & test scripts

[ ] Variables
 [ ]   declaration and allocation
 [ ]   typing variable access expressions
 [ ]   typing assignment to variable expressions
 [ ]   code generation
 [ ]   tests

[ ] If statements
 [ ]   type checking & SAST
 [ ]   code generation
 [ ]   tests

[ ] For loops
 [ ]   formal semantics
 [ ]   type checking & SAST
 [ ]   code generation
 [ ]   tests

[ ] While loops
 [ ]   type checking & SAST
 [ ]   code generation
 [ ]   tests

[ ] Jumps
 [ ]   break semantics & code generation
 [ ]   continue semantics & code generation
 [ ]   return semantics & type checking
 [ ]   return SAST & code generation
 [ ]   tests

[ ] Functions
 [ ]   arguments type checking
 [ ]   return value type checking
 [ ]   SAST
 [ ]   code generation
 [ ]   tests


**Tentative below**
(Tasks with an asterisk are deemed "optional")


[ ] Lists
 [ ]   typing expressions with lists
 [ ]   internal representation, declaration and allocation
 [ ]   built-in functions for lists (part 1, empty?, add)
 [ ]   built-in functions for lists (part 2, first, rest)
 [ ]   strings
 [ ]   built-in functions for strings (print_str)
[ ]   tests

[ ] Objects
 [ ]   declaration & in-memory presentation
 [ ]   assignment to objects (value assignment only, no calling of bound functions)
 [ ]   accessing properties
 [ ]   objects as formal arguments (all pass by reference)
[ ]   tests

[ ] Bindings
 [ ]   extra internal structures for storing the list of bound functions
[ ]   code generation for bind
[ ]   code generation for unbind
[ ]   additional code generation for assignment to property of objects
[ ]*  cycle binding detection
[ ]   tests

[ ]*External objects
[ ]*  alternative code generation for assignment to property of external objects
[ ]   tests

[ ] Runtime environment
[ ]   other built-in functions (math etc)
[ ]   standard libraries
[ ]   basic text-mode runtime
[ ]*  qt based GUI runtime
[ ]*  runtime for the sensor example
[ ]   testing example code & additional tests
\end{verbatim}

Tasks were split among the team members according to this list, but were not
strictly enforced or adhered to.

\subsection{Member roles}

As shown in the authors list, we assigned a role to each member of the team:

\begin{itemize}
\item Tester: Isra Ali
\item Manager: Gwendolyn Edgar
\item Language Guru: Randy Price
\item System Architect: Chris Xiong
\end{itemize}

The roles assigned came from the recommendation of the instructor and were assigned
according to each member's preference. However, these assignments ended up being
rather arbitrary - team members ended up working on their own modules individually,
and became responsible for the modules' design and implementation. Important design
choices were discussed in the Discord server before any final decisions were made.

\subsection{Development environment}

The project is developed and tested on:

\begin{itemize}
\item Gentoo Linux ~amd64, OCaml 4.14.0, LLVM 14.0.1 with OCaml binding installed with system package manager
\item Ubuntu 20.04 LTS on Windows Subsystem for Linux, Windows 10, OCaml 4.13.1, opam 2.1.0, LLVM 10.0.0
\item MacOS Monterey, M1 chip, ocaml 4.13.1, opam 2.1.2, llvm 13.0.1
\item Kali GNU/Linux 2021.4, 5.10.16.3-microsoft-standard-WSL2 , Windows 10, OCaml 4.13.1, opam 2.1.2, LLVM 13.01
\end{itemize}

Git is the source code version control used. All documentation is typeset with \LaTeX .

\subsection{Project log}

See Appendix A for the full git log.
