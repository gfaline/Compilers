\section{Appendix B: Example Programs}

\subsection{Binding Demo}

\subsubsection{Propeller source code (\texttt{binding.pr})}

\begin{mylisting}
objdef Jumbo
{
  str name;
  int age;
  float gpa;
}

fn celebrate(int old, int new) -> void
{
  print(new);
}

fn init () -> int
{
  Jumbo jim;

  jim.name = 'Jim';
  jim.age  = 24;
  jim.gpa  = 3.73;

  bind(jim.age, celebrate);

  jim.age = 25;

  unbind(jim.age, celebrate);

  jim.age = 26;

  return 0;
}
\end{mylisting}

\subsubsection{Generated LLVM IR (\texttt{binding.ll})}

\begin{lstlisting}[language=llvm,backgroundcolor=\color{backgroundcolor}]
; ModuleID = 'Propeller'
source_filename = "Propeller"

@fmt = private unnamed_addr constant [4 x i8] c"%d\0A\00", align 1
@fmt.1 = private unnamed_addr constant [3 x i8] c"%s\00", align 1
@fmt.2 = private unnamed_addr constant [4 x i8] c"%f\0A\00", align 1
@str = private unnamed_addr constant [4 x i8] c"Jim\00", align 1
@fmt.3 = private unnamed_addr constant [4 x i8] c"%d\0A\00", align 1
@fmt.4 = private unnamed_addr constant [3 x i8] c"%s\00", align 1
@fmt.5 = private unnamed_addr constant [4 x i8] c"%f\0A\00", align 1

declare i32 @printf(i8*, ...)

define i32 @init() {
entry:
  %jim = alloca { i8*, i32, double }, align 8
  %jim__name = getelementptr inbounds { i8*, i32, double }, { i8*, i32, double }* %jim, i32 0, i32 0
  store i8* getelementptr inbounds ([4 x i8], [4 x i8]* @str, i32 0, i32 0), i8** %jim__name, align 8
  %jim__age = getelementptr inbounds { i8*, i32, double }, { i8*, i32, double }* %jim, i32 0, i32 1
  store i32 24, i32* %jim__age, align 4
  %jim__gpa = getelementptr inbounds { i8*, i32, double }, { i8*, i32, double }* %jim, i32 0, i32 2
  store double 3.730000e+00, double* %jim__gpa, align 8
  %jim__age1 = getelementptr inbounds { i8*, i32, double }, { i8*, i32, double }* %jim, i32 0, i32 1
  store i32 25, i32* %jim__age1, align 4
  call void @celebrate(i32 25, i32 25)
  %jim__age2 = getelementptr inbounds { i8*, i32, double }, { i8*, i32, double }* %jim, i32 0, i32 1
  store i32 26, i32* %jim__age2, align 4
  ret i32 0
}

define void @celebrate(i32 %old, i32 %new) {
entry:
  %old1 = alloca i32, align 4
  store i32 %old, i32* %old1, align 4
  %new2 = alloca i32, align 4
  store i32 %new, i32* %new2, align 4
  %new3 = load i32, i32* %new2, align 4
  %print = call i32 (i8*, ...) @printf(i8* getelementptr inbounds ([4 x i8], [4 x i8]* @fmt.3, i32 0, i32 0), i32 %new3)
  ret void
}
\end{lstlisting}

\subsubsection{Sensor Demo}

\subsubsection{Propeller source code (\texttt{sensor.pr})}

\begin{mylisting}
# compile with ./prc.sh -r sensor_linux sensor.pr

external objdef Sensor
{
  int temperature;
}

fn print_warning(int oldt, int t) -> void
{
  if (t > 60000) and (t != oldt)
  {
    prints('thermal zone sensor readout too high: ');
    print(t / 1000);
  }
}

fn init() -> int
{
  Sensor sensor;
  bind(sensor.temperature, print_warning);
  return 0;
}
\end{mylisting}

\subsubsection{Generated LLVM IR (\texttt{sensor.ll})}

\begin{lstlisting}[language=llvm,backgroundcolor=\color{backgroundcolor}]
; ModuleID = 'Propeller'
source_filename = "Propeller"

@fmt = private unnamed_addr constant [4 x i8] c"%d\0A\00", align 1
@fmt.1 = private unnamed_addr constant [3 x i8] c"%s\00", align 1
@fmt.2 = private unnamed_addr constant [4 x i8] c"%f\0A\00", align 1
@fmt.3 = private unnamed_addr constant [4 x i8] c"%d\0A\00", align 1
@fmt.4 = private unnamed_addr constant [3 x i8] c"%s\00", align 1
@fmt.5 = private unnamed_addr constant [4 x i8] c"%f\0A\00", align 1
@str = private unnamed_addr constant [39 x i8] c"thermal zone sensor readout too high: \00", align 1

declare i32 @printf(i8*, ...)

define i32 @init() {
entry:
  %sensor = alloca i32, align 4
  %created = call i32 @object_new_Sensor()
  store i32 %created, i32* %sensor, align 4
  %sensor1 = load i32, i32* %sensor, align 4
  call void @object_prop_bind_Sensor_temperature(i32 %sensor1, void (i32, i32)* @print_warning)
  ret i32 0
}

define void @print_warning(i32 %oldt, i32 %t) {
entry:
  %oldt1 = alloca i32, align 4
  store i32 %oldt, i32* %oldt1, align 4
  %t2 = alloca i32, align 4
  store i32 %t, i32* %t2, align 4
  %t3 = load i32, i32* %t2, align 4
  %tmp = icmp sgt i32 %t3, 60000
  %t4 = load i32, i32* %t2, align 4
  %oldt5 = load i32, i32* %oldt1, align 4
  %tmp6 = icmp ne i32 %t4, %oldt5
  %tmp7 = and i1 %tmp, %tmp6
  br i1 %tmp7, label %if, label %else

merge:                                            ; preds = %else, %if
  ret void

if:                                               ; preds = %entry
  %print = call i32 (i8*, ...) @printf(i8* getelementptr inbounds ([3 x i8], [3 x i8]* @fmt.4, i32 0, i32 0), i8* getelementptr inbounds ([39 x i8], [39 x i8]* @str, i32 0, i32 0))
  %t8 = load i32, i32* %t2, align 4
  %tmp9 = sdiv i32 %t8, 1000
  %print10 = call i32 (i8*, ...) @printf(i8* getelementptr inbounds ([4 x i8], [4 x i8]* @fmt.3, i32 0, i32 0), i32 %tmp9)
  br label %merge

else:                                             ; preds = %entry
  br label %merge
}

declare i32 @object_new_Sensor()

declare void @object_prop_bind_Sensor_temperature(i32, void (i32, i32)*)
\end{lstlisting}

\subsubsection{C Runtime source code (\texttt{sensor\_linux.c})}

\begin{lstlisting}[language=C,backgroundcolor=\color{backgroundcolor}]
#include <stdio.h>
#include <unistd.h>

extern int init();

void(*boundf)(int,int) = NULL;
int oldt = 0;

int object_new_Sensor()
{
	return 0;
}

int object_prop_bind_Sensor_temperature(int oid, void(*f)(int,int))
{
	if (oid == 0)
		boundf = f;
}

int object_prop_unbind_Sensor_temperature(int oid, void(*f)(int,int))
{
	if (oid == 0 && boundf == f)
		boundf = NULL;
}

int object_prop_get_Sensor_temperature(int oid)
{
	if (oid == 0)
	{
		return oldt;
	}
	return 0;
}

int main()
{
	int ret = init();
	while (1)
	{
		FILE *f = fopen("/sys/class/thermal/thermal_zone0/temp", "r");
		int t;
		fscanf(f, "%d", &t);
		if (boundf) boundf(oldt, t);
		oldt = t;
		fclose(f);
		sleep(1);
	}
	return ret;
}
\end{lstlisting}
