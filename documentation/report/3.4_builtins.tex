\subsection{Built-in Functions and Runtimes}

Propeller has a minimal set of built-in functions for printing to the standard output:

\begin{tabular}{|l|l|}
\hline
\verb|print(int a)| & Prints an integer value to the console and starts a new line \\\hline
\verb|printb(bool a)| & Prints a boolean value to the console and starts a new line \\\hline
\verb|printf(float a)| & Prints a float value to the console and starts a new line \\\hline
\verb|prints(str a)| & Prints a string to the console\\
\hline
\end{tabular}

Propeller comes with two different runtimes: A default one that just runs the entry function
and does nothing interesting, and one that monitors the ACPI thermal zone of a computer.

The sensor library provides the following object definition:

\begin{mylisting}
external objdef Sensor
{
  int temperature;
}
\end{mylisting}

The value stored in \verb|temperature| is the value read from the temperature sensor in
1/1000 of a degree Celsius. Because a Sensor is an external object, its temperature
field cannot be programatically assigned a value. Instead, its value is updated every second
in the runtime environment. Due to the limitation of the implementation, only one function
can be bound to this property.

The sensor runtime only works on Linux with ACPI thermal zone support (the file \\
\texttt{/sys/class/thermal/thermal\_zone0/temp} must exist on the system).
