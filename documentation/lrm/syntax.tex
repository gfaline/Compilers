\section{Syntax}

\subsection{Comment}
Anything following a hash (\verb|#|) that is not part of a string literal will be treated as a
comment.

\subsection{Literals}
\begin{verbatim}
            <digit> ::= "0"-"9"
      <int-literal> ::= {<digit>}
    <float-literal> ::= {<digit>} "." {<digit>}
     <bool-literal> ::= "true" | "false"
<string-characters> ::= character or escape sequence
   <string-literal> ::= "'" {<string-characters>} "'"
\end{verbatim}

For strings, escape sequences are interpreted the same way OCaml interprets them (as in
\verb|Scanf.unescaped|).

\subsection{Identifiers}
\begin{verbatim}
    <letter> ::= "A"-"Z" | "a"-"z"
<identifier> ::= <letter>
               | <letter> {<letter> | <digit> | "_"} (<letter> | <digit> | "?")
\end{verbatim}

In addition to the rules listed above, identifiers cannot contain two consecutive underscores,
or end with \verb|_?|.

\subsection{Expressions}
\begin{verbatim}
<binary-operators> ::= "+" | "-" | "*" | "/" | "%"
                     | "==" | "!=" | "<" | "<=" | ">" | ">="
                     | "and" | "xor" | "or"
 <unary-operators> ::= "-" | "not"
       <expr-list> ::= <expression> | <expr-list> "," <expression>
        <arg-list> ::= "" | <expr-list>
      <expression> ::= <int-literal>
                     | <float-literal>
                     | <bool-literal>
                     | <string-literal>
                     | <list>
                     | <expression> <binary-operators> <expression>
                     | <unary-operators> <expression>
                     | <identifier> "=" <expression>
                     | "(" <expression> ")"
                     | <identifier> "(" <arg-list> ")"
                     | <identifier> "." <identifier>
                     | <identifier> "." <identifier> = <expression>
\end{verbatim}

Operator precedence follows normal conventions.

\subsection{Lists}
\begin{verbatim}
 <list> ::= "[" arg-list "]"
\end{verbatim}

\subsection{Declarations}

\subsubsection{Variable Declaration}

\begin{verbatim}
     <primitive-type> ::= "int" | "float" | "bool" | "str" | "void"
      <compound-type> ::= "obj" | <primitive-type> "list"
      <variable-decl> ::= <compound-type> <identifier> ";"
 <variable-decl-list> ::= "" | <variable-decl-list> <variable-decl>
\end{verbatim}

\subsubsection{Function Declaration}

\begin{verbatim}
     <formal-list> ::= <compound-type> <identifier>
                     | <formal-list> "," <compound-type> <identifier>
 <formal-list-opt> ::= "" | <formal-list>
   <function-decl> ::= "fn" <identifier> "(" <formal-list-opt> ")" "->"
                       <compound-type> "{" <variable-decl-list>
                       <statement-list> "}"
\end{verbatim}

\subsubsection{Object Type Declaration}

\begin{verbatim}
          <obj-def> ::= "objdef" <identifier> "{" <variable-decl-list> "}"
 <external-obj-def> ::= "external" "objdef" <identifier> "{" <variable-decl-list> "}"
\end{verbatim}

\subsection{Statements}

\subsubsection{Sequencing}

\begin{verbatim}
 <statement-list>  ::= "" | <statement-list> <statement>
 <statememt-block> ::= "{" <statement-list> "}"
\end{verbatim}

\subsubsection{Control Flow}

\begin{itemize}
\item Branching
\begin{verbatim}
    <if-statement> ::= "if" <expression> <statement-block>
                       <elif-statements> <else-clause>
 <elif-statements> ::= {"elif" <expression> <statement-block>}
     <else-clause> ::= "" | "else" <statement-block>
\end{verbatim}
Else clauses are attached to the closest unmatched if clause before it.

\item Loops
\begin{verbatim}
   <for-statement> ::= "for" <identifier> "from" <expression>
                       "to" <expression> <statement-block>
 <while-statement> ::= "while" <expression> <statement-block>
\end{verbatim}

\item Jumps
\begin{verbatim}
    <break-statement> ::= "break" ";"
 <continue-statement> ::= "continue" ";"
   <return-statement> ::= "return" ("" | <expression>) ";"
\end{verbatim}
\end{itemize}

\subsubsection{Special Statements}

Bind and unbind has special syntactical rules. These rules exist only to simplify the
implementation. For users of Propeller they can be treated as if they are normal
built-in functions.

\begin{verbatim}
   <bind-statement> ::= "bind" "(" <identifier> "." <identifier> ","
                        <identifier> ")" ";"
 <unbind-statement> ::= "unbind" "(" <identifier> "." <identifier> ","
                        <identifier> ")" ";"
\end{verbatim}

\subsubsection{All Statements}

\begin{verbatim}
 <statement> ::= <expression> ";"
               | <if-statement>
               | <for-statement>
               | <while-statement>
               | <break-statement>
               | <continue-statement>
               | <return-statement>
               | <bind-statement>
               | <unbind-statement>
\end{verbatim}


% vim: tw=100 spell spelllang=en_us
