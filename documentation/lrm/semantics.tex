\section{Semantics}

Propeller is heavily inspired by languages in the C family, including its semantics. For the sake of
brevity, semantics that are commonly found in other widely-adopted languages will be omitted.

\subsection{Data Types}

Primitive types use the following internal representations:
\begin{center}
\begin{tabular}{| c | c | p{8cm} | }
\hline
 \textbf{Type} & \textbf{Size (bytes)} & \textbf{Description} \\
 \hline
 \texttt{int} & 4 & Integer \\
 \hline
 \texttt{float} & 8 & IEEE 754 floating point \\
 \hline
 \texttt{bool} & 1 & Boolean \\
 \hline
 \texttt{str} & varies & Syntactic sugar for integer lists; stores UTF-8 encoded characters of a
 string \\
\hline
\end{tabular}
\end{center}

Propeller has a list type. Lists are immutable. All items in one list must be of the same type.
List members are stored sequentially in memory.

Custom types can be defined. They can have a number of properties in them. A runtime can have
several predefined custom types, which still must be declared in the program before use, but
prefixed by the \verb|external| keyword.

\subsection{Operations}

In binary operations of numbers, if one of the numbers is a float and the other is an integer, the
integer will be automatically promoted to float. Division between integers gives the integer portion
of the quotient. The modulo operation doesn't support float operands.

Lists can be indexed using the \verb|[]| operator.

\begin{mylistingn}
# comparison
2.3 == 1;   # false
4 != 5;     # true
8.34 > 3;   # true
9 < 10;     # true
5.5 >= 5.6; # false
100 <= 100; # true

# arithmetic
4 + 3;    # 7
2.0 - 1;  # 1.0
-3.3 * 3; # -9.9
5 / 2;    # 2
5.0 / 2;  # 2.5
6 % 4;    # 2

# boolean comparison
true != false; # true
false == true; # false

# boolean operations
not false;      # true
true and false; # false
true or false;  # true
true xor true;  # false
\end{mylistingn}

\begin{mylistingn}
# list indexing
int list l = [1, 2, 3];
l[1];           # 2
\end{mylistingn}

\subsection{Statements}

Most of the control flow in Propeller works like their C counterpart.

For loops use a different syntax and semantics in Propeller. The loop variable is initialized to the
value the first expression evaluates to, is incremented by 1 after the body of the loop is executed
each time, and terminates when the loop variable is greater than the value the second expression
evaluates to. The second expression is reevaluated every time in this process.

\subsection{Binding}

Property of objects can be bound to functions, so that whenever the value of this property
changes, functions bound to that property are called.

Let $\beta$ be the bindings that are currently established during execution of the program. $\beta$
is one of the environment metavariable of Propeller's operational semantics.
$\beta(o, p)$ is a set of functions bound to property $p$ of object $o$. Note that this way objects
of the same custom type don't share bindings.

Function bound to a property must accept 3 parameters: two values of the same type as the property
itself, passing the old value of the property and new value of the property, and one of the object's
type, which will be set to the object whose property value has changed.

When multiple functions are bound to the same property of an object, their order of execution is not
defined and depends on the implementation.

Semi-formal operational semantics of syntactical forms related to binding will be given below.
$\rho(o, p)$ retrieves the location where property $p$ of object $o$ is stored, and $\sigma(l)$ is
the value at location $l$.

$$
\dfrac{\begin{gathered}
\langle e,\rho,\sigma,\beta,\cdots \rangle \Downarrow
\langle v,\rho,\sigma_0,\beta,\cdots \rangle\\
\textrm{for each } f_i\in\beta(o, p), i=1\dots n\\
\langle f_i(\sigma_0(\rho(o,p),v,o),\rho,\sigma_{i-1},\beta,\cdots \rangle \Downarrow
\langle void,\rho,\sigma_i,\beta,\cdots \rangle
\end{gathered}
}
{\langle \textsc{PropertyAssign}(o,p,e),\rho,\sigma,\beta,\cdots \rangle
\Downarrow
\langle v,\rho,\sigma_n\{\rho(o,p)\mapsto v\},\beta,\cdots \rangle
} \qquad \textsc{PropertyAssign}
$$

$$
\dfrac{}
{\langle \textsc{bind}(o,p,f),\rho,\sigma,\beta,\cdots \rangle
\Downarrow
\langle void,\rho,\sigma,\beta\{(o,p)\mapsto \beta(o,p)\cup \{f\}\},\cdots \rangle
} \qquad \textsc{Bind}
$$

$$
\dfrac{}
{\langle \textsc{unbind}(o,p,f),\rho,\sigma,\beta,\cdots \rangle
\Downarrow
\langle void,\rho,\sigma,\beta\{(o,p)\mapsto \beta(o,p)\setminus \{f\}\},\cdots \rangle
} \qquad \textsc{Unbind}
$$

(Note: If the object is of an external type in \textsc{PropertyAssign}, the actual behavior will
differ a little, but that is the case only due to technical reasons and has no difference
semantically.)

\subsection{Program Execution}

When a program written in Propeller is executed, it will start from a function called \verb|init()|.
After \verb|init()| returns, it enters an event loop defined by the runtime library. For the
most basic text-mode only runtime, the event loop simply terminates the program.

% vim: tw=100 spell spelllang=en_us
