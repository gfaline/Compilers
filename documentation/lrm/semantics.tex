\section{Semantics}


% REWORD
% Propeller is heavily inspired by languages in the C family, including its semantics. For the sake of
% brevity, semantics that are commonly found in other widely-adopted languages will be omitted.
% - 

Propeller's operational semantics is heavily inspired by C-like languages. For the sake of
brevity, commonplace semantics found in other widely-used languages are omitted in
favor of the semantics guiding Propeller's most prominent features.

% REORGANIZE
% - this really should be split into three subsubsections, and more detail should be given when describing
%   lists and objects

\subsection{Data Types}

\subsubsection{Primitive Types}
Primitive types use the following internal representations:
\begin{center}
\begin{tabular}{| c | c | p{8cm} | }
\hline
 \textbf{Type} & \textbf{Size (bytes)} & \textbf{Description} \\
 \hline
 \texttt{int} & 4 & Integer \\
 \hline
 \texttt{float} & 8 & IEEE 754 floating point \\
 \hline
 \texttt{bool} & 1 & Boolean \\
 \hline
 \texttt{str} & varies & Syntactic sugar for integer lists; stores UTF-8 encoded characters of a
 string \\
\hline
\end{tabular}
\end{center}

\subsubsection{Lists}
\texttt{list}s contain zero or more elements of the same type, are immutable, and can be
indexed with separators \textbf{[ ]}. Elements of a list are stored sequentially in memory.
For example, a \texttt{bool list} of length 5 takes up 5 bytes in memory, with the first element
stored in the first byte, the second element in the second byte, and so on.

\subsubsection{Objects}
% REWORD
% Custom types can be defined. They can have a number of properties in them. A runtime can have
% several predefined custom types, which still must be declared in the program before use, but
% prefixed by the \verb|external| keyword.
% - reworded for better flow and clarity
Propeller allows users to define custom types called \textit{objects}. An object has one or
more properties, which are variables of primitive types. Additionally, functions may be bound
to these properties and executed upon a change in the property's value. A runtime can have
several predefined objects from an external library, but these objects must be defined
and prefixed with the \texttt{external} keyword

\subsection{Operations}

% REWORD
% In binary operations of numbers, if one of the numbers is a float and the other is an integer, the
% integer will be automatically promoted to float. Division between integers gives the integer portion
% of the quotient. The modulo operation doesn't support float operands.
% - reworded to be more precice
% - added detail about int / int = float

\subsubsection{\texttt{int} and \texttt{float} Operations}
Comparison and arithmetic operators are overloaded in Propeller. Comparing an \texttt{int} to a
\texttt{float} will result in the promotion of the \texttt{int} expression to a \texttt{float} expression;
similarly, combining an \texttt{int} and a \texttt{float} with a binary arithmetic operator will result
in the promotion of the \texttt{int} expression to a \texttt{float} expression, and the result of the
arithmetic operation will be a \texttt{float}. Additionally, the modulus \texttt{\%}
operator, only takes \texttt{int} operands, and the divison of an \texttt{int} by a \texttt{int} returns
a \texttt{float}.

\begin{mylisting}
# comparison
2.3 == 1;   # false
4 != 5;     # true
8.34 > 3;   # true
9 < 10;     # true
5.5 >= 5.6; # false
100 <= 100; # true

# arithmetic
4 + 3;    # 7
2.0 - 1;  # 1.0
-3.3 * 3; # -9.9
5 / 2;    # 2
5.0 / 2;  # 2.5
6 % 4;    # 2

# boolean comparison
true != false; # true
false == true; # false

# boolean operations
not false;      # true
true and false; # false
true or false;  # true
true xor true;  # false
\end{mylisting}

\noindent Lists can be indexed using the \verb|[]| operator.

\begin{mylisting}
# list indexing
int list l = [1, 2, 3];
l[1];           # 2
\end{mylisting}

% Renamed to control flow
\subsection{Control Flow}
Propeller uses \texttt{if/elif/else} clauses, \texttt{for} loops, and \texttt{while} loops.
Each control flow method in its entirety is a statement, and the body of an
\texttt{if/elif/else} clause or loop is comprised of a list of statements. Control flow
semantics are C-like, with the following differences:
\begin{itemize}
    \item \texttt{if/elif/while} expressions need not be enclosed in parentheses.
    \item Statements following \texttt{if/elif/else}, \texttt{for}, and \texttt{while} must
          be enclosed in curly braces \texttt{\{ \}}.
    \item There is no such thing as a "block" - each control flow method is followed by a list
          of one or more statements enclosed in curly braces.
    \item \texttt{for} loops have a unique syntax, and are intended to execute a given number of
          times. When writing a for loop, the name of the looping variable must be given, along
          with integer expressions that evaluate to the looping variable's initial and final
          value, respectively. When the looping variable is greater than the final value, the
          loop terminates.
\end{itemize}

% REWORD
% Most of the control flow in Propeller works like their C counterpart.

% For loops use a different syntax and semantics in Propeller. The loop variable is initialized to the
% value the first expression evaluates to, is incremented by 1 after the body of the loop is executed
% each time, and terminates when the loop variable is greater than the value the second expression
% evaluates to. The second expression is reevaluated every time in this process.
% - the description above is more consise and clear

\subsection{Binding}

Property of objects can be bound to functions, so that whenever the value of this property
changes, functions bound to that property are called.

Let $\beta$ be the bindings that are currently established during execution of the program. $\beta$
is one of the environment metavariable of Propeller's operational semantics.
$\beta(o, p)$ is a set of functions bound to property $p$ of object $o$. Note that this way objects
of the same custom type don't share bindings.

Function bound to a property must accept 3 parameters: two values of the same type as the property
itself, passing the old value of the property and new value of the property, and one of the object's
type, which will be set to the object whose property value has changed.

When multiple functions are bound to the same property of an object, their order of execution is not
defined and depends on the implementation.

Semi-formal operational semantics of syntactical forms related to binding will be given below.
$\rho(o, p)$ retrieves the location where property $p$ of object $o$ is stored, and $\sigma(l)$ is
the value at location $l$.

$$
\dfrac{\begin{gathered}
\langle e,\rho,\sigma,\beta,\cdots \rangle \Downarrow
\langle v,\rho,\sigma_0,\beta,\cdots \rangle\\
\textrm{for each } f_i\in\beta(o, p), i=1\dots n\\
\langle f_i(\sigma_0(\rho(o,p),v,o),\rho,\sigma_{i-1},\beta,\cdots \rangle \Downarrow
\langle void,\rho,\sigma_i,\beta,\cdots \rangle
\end{gathered}
}
{\langle \textsc{PropertyAssign}(o,p,e),\rho,\sigma,\beta,\cdots \rangle
\Downarrow
\langle v,\rho,\sigma_n\{\rho(o,p)\mapsto v\},\beta,\cdots \rangle
} \qquad \textsc{PropertyAssign}
$$

$$
\dfrac{}
{\langle \textsc{bind}(o,p,f),\rho,\sigma,\beta,\cdots \rangle
\Downarrow
\langle void,\rho,\sigma,\beta\{(o,p)\mapsto \beta(o,p)\cup \{f\}\},\cdots \rangle
} \qquad \textsc{Bind}
$$

$$
\dfrac{}
{\langle \textsc{unbind}(o,p,f),\rho,\sigma,\beta,\cdots \rangle
\Downarrow
\langle void,\rho,\sigma,\beta\{(o,p)\mapsto \beta(o,p)\setminus \{f\}\},\cdots \rangle
} \qquad \textsc{Unbind}
$$

% REWORD
% \noindent If the object is of an external type in \textsc{PropertyAssign}, the actual behavior will
% differ a little, but that is the case only due to technical reasons and has no difference
% semantically.
% - grammar/readbility
Note that if the object in \textsc{PropertyAssign} is defined as \texttt{external},
the actual behavior will differ slightly, but will be semantically indistinguishable from a
non-\texttt{external} object.

\subsection{Program Execution}

When a program written in Propeller is executed, it will start from a function called \verb|init()|.
After \verb|init()| returns, it enters an event loop defined by the runtime library. For the
most basic text-mode only runtime, the event loop simply terminates the program.

% vim: tw=100 spell spelllang=en_us
